\documentclass[12pt,a4paper]{article}
\usepackage[utf8]{inputenc}
\usepackage[T1]{fontenc}
\usepackage[margin=2.5cm]{geometry}
\usepackage{xcolor}
\usepackage{enumitem}
\usepackage{fancyhdr}
\usepackage{titlesec}
\usepackage[most]{tcolorbox}
\usepackage{graphicx}
\usepackage{hyperref}

% Colors
\definecolor{navyblue}{HTML}{003366}
\definecolor{ieltsred}{HTML}{C8102E}
\definecolor{correctgreen}{HTML}{28a745}
\definecolor{lightgray}{HTML}{f8f9fa}
\definecolor{tipyellow}{HTML}{fff3cd}

% Header and footer
\pagestyle{fancy}
\fancyhf{}
\setlength{\headheight}{15pt}
\lhead{\textcolor{navyblue}{IELTS Writing -- Good Answer Examples}}
\rhead{\textcolor{navyblue}{\today}}
\cfoot{\thepage}

% Section formatting
\titleformat{\section}
{\Large\bfseries\color{navyblue}}
{\thesection}{1em}{}

\titleformat{\subsection}
{\large\bfseries\color{navyblue}}
{\thesubsection}{1em}{}

% Image path
\graphicspath{{../../media/ielts-writing-samples/}}

% Custom boxes
\newtcolorbox{taskbox}[1][]{
    enhanced,
    colback=white,
    colframe=navyblue,
    colbacktitle=navyblue,
    coltitle=white,
    fonttitle=\bfseries,
    title=#1,
    breakable,
    pad at break*=1mm,
    segmentation style={solid, draw=gray},
    before skip=0.5\baselineskip plus 2pt,
    after skip=0.5\baselineskip plus 2pt
}

\newtcolorbox{instructionbox}{
    colback=lightgray,
    colframe=gray,
    fonttitle=\bfseries,
    title={\textbf{Task Instructions}},
    breakable,
    pad at break*=1mm,
    segmentation style={solid, draw=gray},
    before skip=0.5\baselineskip plus 2pt,
    after skip=0.5\baselineskip plus 2pt
}

\newtcolorbox{modelanswer}{
    enhanced,
    colback=green!5!white,
    colframe=correctgreen,
    fonttitle=\bfseries,
    title={\textbf{Model Answer (Band 8--9)}},
    breakable,
    pad at break*=1mm,
    segmentation style={solid, draw=gray},
    before skip=0.5\baselineskip plus 2pt,
    after skip=0.5\baselineskip plus 2pt
}

\begin{document}

\begin{center}
    {\Huge\bfseries\textcolor{navyblue}{IELTS Writing}}\\[0.3cm]
    {\LARGE\textcolor{ieltsred}{Good Answer Examples}}\\[0.5cm]
    {\large A collection of model answers for practice and reference}\\[0.3cm]
    \rule{0.8\textwidth}{0.4pt}
\end{center}

\tableofcontents
\newpage

%=============================================================================
\section{Writing Task 1 -- Academic}
%=============================================================================

Task 1 of the Academic IELTS requires you to describe visual information (graphs, charts, diagrams, maps, or processes).

\subsection{Bar Charts}

%-----------------------------------------------------------------------------
\subsubsection{Example 1: Cuppa Coffee Shop Sales}
%-----------------------------------------------------------------------------

\begin{taskbox}[Task Question]

The chart below shows the sales for a coffee shop in a town in the UK from 2000 to 2020.

\begin{center}
\includegraphics[width=0.67\textwidth]{cuppa-coffee-shop-sales.png}
\end{center}
\end{taskbox}

\begin{instructionbox}
\textit{Summarise the information by selecting and reporting the main features, and make comparisons where relevant.}
\end{instructionbox}

\begin{modelanswer}
The bar chart illustrates the typical monthly revenue of Cuppa Coffee Shop, located in a British town, from the sales of six food and drink items in 2000, 2010, and 2020. Units are measured in pounds sterling.
\end{modelanswer}

\subsection{Table and Pie Charts}

%-----------------------------------------------------------------------------
\subsubsection{Example 1: US Fish Imports}
%-----------------------------------------------------------------------------

\begin{taskbox}[Task Question]
The table and charts below give information about the total cost and sources of fish imported to the US between 1988 and 2000.

\begin{center}
\includegraphics[width=0.6\textwidth]{us-fish-imports-table-pie-charts.png}
\end{center}
\end{taskbox}

\begin{instructionbox}
\textit{Summarise the information by selecting and reporting the main features, and make comparisons where relevant.}
\end{instructionbox}

\begin{modelanswer}
The table shows the total expenditure of the US, in billions of dollars, on imported fish, in 1988, 1992 and 2000, while the three pie charts illustrate the proportion of fish that was sourced from China, Canada and other countries in the same three years.
\end{modelanswer}

\newpage

% ===========================
% EXAMPLE 3: MAPS
% ===========================
\subsection{Maps}

Maps are a common task type in IELTS Writing Task 1. Understanding the different types and appropriate grammar structures is essential for success.

\subsubsection{Types of IELTS Maps}

There are a variety of maps that you can get in IELTS Writing Task 1 test:

\begin{enumerate}
\item \textbf{Changes in Towns}

These types of maps focus on the expansion and layout of towns with features such as roads, buildings, land and other features of a town or city. It is most common to be given two maps, but occasionally (like in the model below) you could get three maps.

\item \textbf{Changes in a Resort}

This is about a holiday area which usually has different features than a town. There might be facilities such as swimming pools, nature hikes, hotels, beaches and water features. These types of tasks often compare current resorts with a future resort. However, future maps could appear as any type of map.

\item \textbf{Places with Multiple Buildings and Features}

Typical examples of this type of map is a school, university or hospital.

This type of map covers an area of ground containing different buildings that serve different functions. The layout is different to a town and the facilities relate only to the function of the company/institution.

\item \textbf{Floor Plans}

You could be given a floor plan which means it is the layout of a building on the inside showing all the rooms. Floor plans often show a past layout with a future plan to expand and alter rooms. Unlike the above maps, this one is about rooms and the functions of rooms. For example a room might have been a study in the past but there are plans to expand it and use it as a kitchen/diner.
\end{enumerate}

\subsubsection{Grammar Tenses for Map Writing}

Always check the date on the maps:

\begin{enumerate}
\item if the map is dated in the \textbf{past}, you must use past tense. For example, ``The hospital was located to the north side of the town''

\item if the map \textbf{shows a future plan}, you must use future forms, such as ``it will be extended and will no longer be used as an office, but instead used as a reception room.''

\item if the map is dated as ``Present'' or ``Now'', you would use the present tense.

\item If there is a \textbf{comparison of dates}, you must be flexible with the tenses in your sentences: ``the office \underline{was located} on the ground floor but in the future it \underline{will be moved} to the first floor.''

\item You will also notice that the passive voice is sometimes used for map reports for writing task 1.
\end{enumerate}

\subsubsection{Comments about Model Answer}

\begin{enumerate}
\item It isn't often that you will have three body paragraphs for your IELTS Writing Task 1 report. But this maps has three time periods so it makes sense to have these body paragraphs.

\item It could be possible to divide the information of body paragraphs into:
\begin{itemize}
\item Body Paragraph 1 -- roads and railway
\item Body Paragraph 2 -- land and buildings
\item There is no right or wrong way to organise information into body paragraphs. You are being marked on being logical in how you organise information. If it lacks logic, you get a lower score. Your organisation also needs to help highlight key features which means deciding key features during your planning state is important because it will influence your paragraphing.
\end{itemize}

\item The length of all writing task 1 should be between 170 and 190 words. A longer report will be marked down for not selecting features and getting lost in detail. A shorter report will be marked for not having enough information.
\end{enumerate}

\subsubsection{Example 1: Meadowside and Fonton Development}

\begin{taskbox}[Task Question]
The maps below show the changes that have taken place in Meadowside village and Fonton, a neighbouring town, since 1962.

\begin{center}
\includegraphics[width=0.45\textwidth]{../../media/ielts-writing-samples/meadowside-fonton-maps.jpg}
\end{center}
\end{taskbox}

\begin{instructionbox}
\textit{Summarise the information by selecting and reporting the main features, and make comparisons where relevant.}
\end{instructionbox}

\begin{modelanswer}
The three maps illustrate how Meadowside village and Fonton, which is a nearby town, have developed from 1962 to the present.\\

Overall, both Fonton and Meadowside village increased in size over the years until they eventually merged together, at which point Meadowside became a suburb. Furthermore, there have been significant changes to infrastructure, housing and facilities over the period given.\\

In 1962, both Meadowside and Fonton were completely separate with no roads or rail connecting them. While Fonton had a railway line running to the north, Meadowside, located to the west of Fonton, only had a small road from the west.\\

By 1985, Meadowside had expanded and the small road had become a main road. A further main road had been built to connect the village to Fonton. Within Meadowside, a superstore, leisure complex and housing estate had been developed. By this time, Fonton had also grown in size.\\

Currently, Meadowside is known as Meadowside Suburbs after joining with Fonton. Between both places, a hotel, station and business park have been built on either side of the railway line.\\
\end{modelanswer}

\vfill
\begin{center}
\rule{0.8\textwidth}{0.4pt}\\[0.3cm]
\textit{Last updated: \today}
\end{center}

\end{document}